\documentclass[12pt]{article}

\title{Chosen sequences among linear congruential generators using SAT/SMT solvers: a Valentine's Day card to Mayflor Jamero Holden}
\author{William John Holden}
\date{\today}

\usepackage[margin=1.0in]{geometry}
\usepackage{listings}
\lstset{basicstyle=\ttfamily,columns=flexible,frame=single,breaklines=true,linewidth=1\textwidth}

\usepackage[hyphens]{url}
\usepackage[hidelinks]{hyperref}

% https://oeis.org/wiki/List_of_LaTeX_mathematical_symbols

\begin{document}
\maketitle

What does the following program do?

\begin{lstlisting}[language=Java]
import java.util.*;

public class RandomMessage {

    private static long[] l = new long[] {
        247057036367419L, 35543432715160L, 253980514236592L,
        89613466789088L, 53058416132968L, 111106152524424L,
        231264589651259L, 218141079436269L, 203524665566275L,
        99273655041031L, 217249958230482L
    };

    public static void main(String args[]) {
        for (int i = 0 ; i < l.length ; i++) {
            Random random = new Random(l[i]);
            System.out.print((char)(random.nextInt() & 95));
            System.out.print((char)(random.nextInt() & 95));
            System.out.print((char)(random.nextInt() & 95));
        }
    }
}
\end{lstlisting}

You probably suspect that it does something unexpected, and of course you are correct.
This program iterates through each long integer listed in the \texttt{l} array, instantiates a new pseudo-random number generator (PRNG) with that value as a \textit{seed}, and prints the PRNG's first three outputs as uppercase letters.

Before we look into the particular output, we should quickly review the internals of the \texttt{java.lang.Random} class.
The \texttt{Random} class implements a linear congruential formula defined by D. H. Lehmer and described by Donald Knuth in \textit{The Art of Computer Programming}.
The OpenJDK implementation for \texttt{Random} can be found at \url{http://hg.openjdk.java.net/jdk10/jdk10/jdk/file/777356696811/src/java.base/share/classes/java/util/Random.java}.
Inside this implementation we see three magical constants (\texttt{multiplier}, \texttt{addend}, and \texttt{mask}) and an instance variable \texttt{seed} of type \texttt{AtomicLong}.

When the constructor for \texttt{Random} is provided a seed value, this initial seed is first scrambled by the \texttt{initialScramble} method.
\texttt{initialScramble} performs a bitwise exclusive or against the \texttt{multiplier} constant, followed by a bitwise and against the \texttt{mask}.
\texttt{Random} provides no hints for the origin of the \texttt{multiplier} value, which is \texttt{0x5DEECE66DL} (decimal 25214903917).

By contrast, the origin for \texttt{mask} is clear.
\texttt{Random} is designed for a 48-bit seed.
\texttt{mask} is initialized with the statement \texttt{(1L << 48) - 1)}, which results in bit string of 16 zeros and 48 ones.

\texttt{Random} contains a protected \texttt{next} method that computes the next seed as \texttt{nextseed = (oldseed * multiplier + addend) \& mask} and returns up to 48 bits of the new seed.
The public \texttt{nextInt} method is implemented as the one-liner \texttt{return next(32)}.

Seeing the beautiful simplicity of this program, can we construct a seed to produce chosen sequences?
Yes.
In Dennis Yurichev's \textit{SAT/SMT by Example}, we see that it is possible for Microsoft Visual C++'s (MSVC) \texttt{rand} function to produce eight consecutive decimal numbers ending with zero.
Moreover, Yurichev shows a method to discover the necessary seed value using Microsoft's Z3 SAT/SMT solver.

The following Python program builds upon Yurichev's previous work and discovers the seed values necessary to recreate a desired string.

\begin{lstlisting}[language=Python]
from z3 import *

def find_message_seed(msg, start):
    # Instantiate a new Z3 solver.
    s = Solver()

    # Java's PRNG uses the following three constants.
    multiplier = BitVec('multiplier', 64)
    mask = BitVec('mask', 64)
    addend = BitVec('addend', 64)
    s.add(multiplier == 0x5DEECE66D, mask == (1 << 48) - 1, addend == 0xB)

    # The seed is the answer we are looking for.
    # To word backwards from our intended message to the seed, the PRNG
    # will have to cycle through one or more states.
    # I kind of doubt we will ever find a lucky enough streak of PRNG
    # results to get past eight states.
    seed = BitVec('seed', 64)
    states = [BitVec("state{n}".format(n = i), 64) for i in range(0,7)]

    # Again, we don't know the seed, but we do not know the seed will be
    # used to generate the initial state of the PRNG. Also, yes the
    # multiplier gets "anded" to the seed, not multiplied.
    # http://hg.openjdk.java.net/jdk10/jdk10/jdk/file/777356696811/src/java.base/share/classes/java/util/Random.java#l145
    s.add(states[0] == (seed ^ multiplier) & mask)

    # Specify state transitions. This reproduces the behavior of
    # http://hg.openjdk.java.net/jdk10/jdk10/jdk/file/777356696811/src/java.base/share/classes/java/util/Random.java#l203
    # Note that the magic value "addend" is added here but was
    # not used to get to state[0] above.
    for j in range(1,7):
        s.add(states[j] == (states[j - 1] * multiplier + addend) & mask)
    
    # Z3 also needs to know that we are trying to relate the PRNG
    # states to letters. For this, we will use up to seven characters.
    # Note that state 0 does not correspond to a character. State 0 is
    # the initalized state that can never be directly used to generate
    # a returnable value from the PRNG.
    characters = [BitVec("ch{n}".format(n = i), 64) for i in range(1,7)]   

    # Ok, we've got our PRNG states and characters. Now we need to marry
    # them somehow. In this program, the relation between a PRNG state
    # and a character is that we:
    # 1) Shift the seed by 16 bits just like java.util.Random.next(32),
    # 2) Extract only the last seven bits from this value using a mask,
    # 3) Always set the fifth bit (0x20) to make the letter uppercase.
    #
    # This program can handle spaces.
    # Leaving the fifth bit free will allow you to work with more 
    # characters, but you might not be able to find solutions for your
    # input.
    #
    # I tried making this an "or" statement so that the solver could
    # branch with the fifth bit set or unset. Bad idea.
    for k in range(0,6):
        s.add(characters[k] == ((states[k + 1] >> 16) & 0x7f) | 0x20)
    
    # Now we can constrain the value for these characters.
    # We walk from left to right, adding constraints letters to our
    # string until the solver determines no such sequence is possible.    
    longest_seed = -1
    i = 0
    # You don't have to constrain i to just 3. In my experience, though,
    # you won't find many outputs that produce more than 3 characters,
    # while you can fairly reliably find three-character sequences.
    # The program completes much faster with this constraint.
    while i < 3 and start + i < len(msg):
        s.add(characters[i] == ord(msg[start + i]))
        if s.check() == sat:
            i = i + 1
            longest_seed = s.model()[seed].as_long()
        else:
            break

    # Return the value of the longest seed as a long with the
    # number of characters constructed.
    return (longest_seed, i)

def find_message_seeds(msg):
    position = 0
    result = []
    while position < len(msg):
        (seed, msg_length) = find_message_seed(msg, position)
        result.append((seed, msg_length))
        position += msg_length
    return result
\end{lstlisting}

So what does the Java program on the first output?

\begin{lstlisting}
$ java RandomMessage
HAPPY VALENTINES DAY MAYFLOR LOVE
\end{lstlisting}

Happy Valentine's Day, Mayflor!

\end{document}